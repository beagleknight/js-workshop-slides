\section{Stage 8: Multiplayer game with Node.js: Express and socket.io}

\begin{frame}[fragile]
  \begin{center}
    %\includegraphics[width=300px]{images/map_stage_6.png}
  \end{center}
\end{frame}

\begin{frame}[fragile]
  \frametitle{Introduction}

  We are going to do our first steps on a multiplayer game using Websockets. First, we need a web server to serve our web application (our game) and handle communications between all clients connected to the server.

  \pause

  \begin{block}{Express.js}
    Express is a minimal and flexible node.js web application framework, providing a robust set of features for building single and multi-page, and hybrid web applications.
  \end{block}
\end{frame}
    
\begin{frame}[fragile]
  \frametitle{Express.js: Hello World!}

  \begin{block}{Installation}
    {\tiny
    \begin{verbatim}
    npm install express --save-dev
    \end{verbatim}
    }
  \end{block}

  \pause

  \begin{block}{Hello World}
    {\tiny
    \begin{verbatim}
    var express = require('express'),
        app     = express();

    app.get('/hello.txt', function(req, res){
        var body = 'Hello World';
        res.setHeader('Content-Type', 'text/plain');
        res.setHeader('Content-Length', body.length);
        res.end(body);
    });

    app.listen(9000);
    console.log('Listening on port 9000');
    \end{verbatim}
    }
  \end{block}

  \pause

  \begin{block}{Run server}
    {\tiny
    \begin{verbatim}
    $ node index.js // Open browser and visit http://localhost:9000
    \end{verbatim}
    }
  \end{block}
\end{frame}

\begin{frame}[fragile]
  \frametitle{Express.js: API}

  \begin{block}{Serving static files with \texttt{app.use}}
    {\tiny
    \begin{verbatim}
    app.use(express.static(__dirname + '/public'));
    \end{verbatim}
    }
  \end{block}

  \pause

  \begin{block}{Using a template engine system}
    Install a template engine system. For example \texttt{jade}:
    {\tiny
    \begin{verbatim}
    $ npm install jade --save-dev
    \end{verbatim}
    }
    Use it on our Express application:
    {\tiny
    \begin{verbatim}
    app.set("view engine", "jade");
    app.set("views", __dirname + "/views");
    \end{verbatim}
    }
  \end{block}

\end{frame}

\begin{frame}[fragile]
  \frametitle{Express.js: API}

  \begin{block}{Defining routes and rendering views}
    {\tiny
    \begin{verbatim}
    app.get('/about', function (req, res) {
        res.render('about');
    });

    app.get('/credits', function (req, res) {
      res.render('credits', { name: 'test' }); // Pass parameters to the view
    });

    app.post('/players', function (req, res) {
        res.render('players/show');
    });

    app.put('/players', function (req, res) {
        res.render('players/show');
    });
    \end{verbatim}
    }
  \end{block}
\end{frame}

\begin{frame}[fragile]
  \frametitle{Express.js: API}

  \begin{block}{Parse query parameters}
  {\tiny
  \begin{verbatim}
  // GET /search?q=nintendo
  app.get('/search', function (req, res) {
      var q = req.query.q;
  });
  \end{verbatim}
  }
  \end{block}

  \pause

  \begin{block}{Parse body}
  First, we need to use \texttt{bodyParser} middleware.
  {\tiny
  \begin{verbatim}
  app.use(express.bodyParser());
  \end{verbatim}
  }
  Then, we can parse body directly:
  {\tiny
  \begin{verbatim}
  // POST /players player[name]=David
  app.post('/players', function (req, res) {
      var playerData = req.body.player,
          playerName = playerData.name;
  });
  \end{verbatim}
  }

  \end{block}
\end{frame}

\begin{frame}[fragile]
  \frametitle{Express.js: Lab}

  \begin{itemize}
    \item \texttt{git checkout stage\_8\_1}
    \item Install your back-end dependencies with \texttt{npm install}
    \item Install your front-end dependencies with \texttt{bower install}
    \item Start \texttt{grunt watch} for auto linting
    \item Find TODOs and complete the exercise.
  \end{itemize}
\end{frame}
