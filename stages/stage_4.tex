\section{Stage 4: Automate your workflow with Grunt}

\begin{frame}[fragile]
\end{frame}

\begin{frame}[fragile]
  \frametitle{What is Grunt?}
  \begin{block}{Definition}
    Grunt is a Javascript Task Runner. Simplify your life automating tedious tasks like minification, compilation, unit testing, etc. There are a lot of available Grunt plugins, take a look to the plugin directory at \url{http://gruntjs.com/plugins}.
  \end{block}

  \begin{center}
    \includegraphics[width=100px]{images/grunt.png}
  \end{center}
\end{frame}

\begin{frame}[fragile]
  \frametitle{Installation}

  \begin{block}{Using npm}
  {\scriptsize
    \begin{verbatim}
    $ npm install -g grunt-cli
    \end{verbatim}
  }
  \end{block}

  \pause

  \begin{block}{\texttt{Gruntfile.js} and \texttt{package.json}}
    Grunt needs a \texttt{Gruntfile.js} on your project's directory in order to work. Also, we need a \texttt{package.json} (similar to our \texttt{bower.json} but for our server-side dependencies) to add our Grunt plugins as a dependencies for our project.
  {\scriptsize
    \begin{verbatim}
    $ npm init
    // Default answers
    $ npm install grunt --save-dev
    \end{verbatim}
  }
  \end{block}
\end{frame}

\begin{frame}[fragile]
  \frametitle{Gruntfile.js}

  \begin{block}{Basic Gruntfile.js}
  {\scriptsize
  \begin{verbatim}
  module.exports = function (grunt) {  
      grunt.registerTask('default', []);
  };
  \end{verbatim}
  }
  \end{block}

  \pause

  \begin{block}{Run default task}
  {\scriptsize
  \begin{verbatim}
  $ grunt
  Done, without errors
  \end{verbatim}
  }
  \end{block}
\end{frame}

\begin{frame}[fragile]
  \frametitle{JSHint}

  JSHint is a tool that helps to detect errors and potential problems in your Javascript code.

  \pause

  \begin{block}{\texttt{grunt-contrib-jshint}}
    {\tiny
    \begin{verbatim}
    $ npm install grunt-contrib-jshint --save-dev
    \end{verbatim}
    }
  \end{block}
\end{frame}

\begin{frame}[fragile]
  \frametitle{JSHint}

  \begin{block}{\texttt{Gruntfile.js}}
    {\tiny
    \begin{verbatim}
    module.exports = function (grunt) {
        grunt.initConfig({
            jshint: {
                all: {
                    options: {
                        jshintrc: '.jshintrc'                                               
                    },
                    files: {
                        src: ['Gruntfile.js', 'js/**/*.js']
                    }
                }
            }
        });

        grunt.loadNpmTasks('grunt-contrib-jshint');
        grunt.registerTask('default', ['jshint']);                                          
    };
    \end{verbatim}
    }
  \end{block}
\end{frame}

\begin{frame}[fragile]
  \frametitle{JSHint}
  \begin{block}{\texttt{.jshintrc}}
    {\tiny
    \begin{verbatim}
      {
        ``curly'': true,
        ``eqeqeq'': true,
        ``immed'': true, 
        ``latedef'': true,
        ``newcap'': true,   
        ``noarg'': true,
        ``sub'': true,  
        ``undef'': true,    
        ``unused'': true,
        ``boss'': true,
        ``eqnull'': true,
        ``browser'': true,
        ``node'': true,
        ``expr'': true,
        ``globals'': {
        } 
      }
    \end{verbatim}
    }
  \end{block}
\end{frame}

\begin{frame}[fragile]
  \frametitle{JSHint}
  \begin{block}{Running task}
  {\scriptsize
    \begin{verbatim}
    $ grunt jshint:all
    Running ``jshint:all'' (jshint) task
    Linting js/main.js ...ERROR
    [L1:C1] W117: '$' is not defined.
    $(function ( {
    [L29:C18] W117: '$' is not defined.
      var el = $(``#'' + character.id);
    [L50:C18] W117: '$' is not defined.
      var el = $("#" + character.id);

    Warning: Task ``jshint:all'' failed. Use --force to continue.
    Aborted due to warnings.
    $ grunt jshint
    // Same output
    $ grunt
    // Same output
    \end{verbatim}
  }
  \end{block}
\end{frame}
