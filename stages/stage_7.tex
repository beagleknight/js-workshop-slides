\section{Stage 7: Events, events everywhere: The Javascript Event Loop}

\begin{frame}[fragile]
  \begin{center}
    %\includegraphics[width=300px]{images/map_stage_6.png}
  \end{center}
\end{frame}

\begin{frame}[fragile]
  \frametitle{Event Loop}
  \begin{block}{Event Loop}
    JavaScript has a concurrency model based on an ``event loop''. This model is quite different than the model in other languages like C or Java.
    {\scriptsize
    \begin{verbatim}
      while(queue.waitForMessage()) {
        queue.processNextMessage();
      }
    \end{verbatim}
    }
    \pause
    A very interesting property of the event loop model is that JavaScript, unlike a lot of other languages, never blocks.
  \end{block}
\end{frame}

\begin{frame}[fragile]
  \frametitle{Event Loop}

  \begin{block}{Example}
    {\tiny
    \begin{verbatim}
    var now = +new Date();

    setTimeout(function () {
        var dt = (+new Date()) - now;
        console.log('First timeout');
        console.log('Elapsed time: ' + dt + ' milliseconds');
    }, 500);

    setTimeout(function () {
        var dt = (+new Date()) - now,
            i = 0;
        console.log('Second timeout');
        console.log('Elapsed time: ' + dt + ' milliseconds');
        while(i < 1000000000) {
          i += 1;
        }
    }, 250);
    \end{verbatim}
    }
  \end{block}

  \pause

  \begin{block}{Output}
    {\tiny
    \begin{verbatim}
    Second timeout
    Elapsed time: 252 milliseconds
    First timeout
    Elapsed time: 1271 milliseconds
    \end{verbatim}
    }
  \end{block}
\end{frame}

\begin{frame}[fragile]
  \frametitle{DOM Events}

  When an event happens, the browser sends the event to the related element. If you've set a handler (a function) on that element, it gets called with related event info which means you 'handled' the event.

  \pause

  \begin{block}{The old way}
    {\scriptsize
    \begin{verbatim}
    var blueSoldierSeat = document.getElementById('blue-soldier-seat');
    blueSoldierSeat.onclick = function (event) {
        // code omitted
    };
    \end{verbatim}
    }
  \end{block}

  \pause

  \begin{block}{The classy way}
    {\scriptsize
    \begin{verbatim}
    var blueSoldierSeat = document.getElementById('blue-soldier-seat');
    blueSoldierSeat.addEventListener('click', function (event) {
        // code omitted
    });
    \end{verbatim}
    }
  \end{block}
\end{frame}

\begin{frame}[fragile]
  \frametitle{DOM Events}

  \begin{block}{The jQuery way \#1}
    {\scriptsize
    \begin{verbatim}
    var $blueSoldierSeat = $('#blue-soldier-seat');
    $blueSoldierSeat.click(function (event) {
        // code omitted
    });
    \end{verbatim}
    }
  \end{block}

  \pause

  \begin{block}{The jQuery way \#2}
    {\scriptsize
    \begin{verbatim}
    var $blueSoldierSeat = $('#blue-soldier-seat');
    $blueSoldierSeat.on('click', function (event) {
        // code omitted
    });
    \end{verbatim}
    }
  \end{block}
\end{frame}

\begin{frame}[fragile]
  \frametitle{Custom Events}

  We can use a library or implement our own functions for listening and triggering custom events.

  \pause

  \begin{block}{With jQuery \texttt{on} and \texttt{trigger}}
    {\scriptsize
    \begin{verbatim}
    var tree = {
        apples: 10
    };

    $(document).on('apple fall', function (event) {
        tree.apples -= 1;
    });

    $(document).trigger('apple fall');

    console.log(tree.apples); // Output: 9
    \end{verbatim}
    }
  \end{block}
\end{frame}

\begin{frame}[fragile]
  \frametitle{Custom Events}
  \begin{block}{With library \texttt{IndigoUnited/events-emitter}}
    {\scriptsize
    \begin{verbatim}
    // Require.js code omitted
    var EventsEmitter  = require('events-emitter/EventsEmitter'),
        emitter        = new EventsEmitter(),
        tree           = {
            apples: 10
        };

    emitter.on('apple fall', function (event) {
        tree.apples -= 1;
    });

    emitter.emit('apple fall');

    console.log(tree.apples); // Output: 9
    \end{verbatim}
    }
  \end{block}
\end{frame}
