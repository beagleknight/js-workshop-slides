\section{Stage 9: Persistance with RedisDB}

\begin{frame}[fragile]
  \begin{center}
    %\includegraphics[width=300px]{images/map_stage_6.png}
  \end{center}
\end{frame}

\begin{frame}[fragile]
  \frametitle{Persistance}

  We stored player positions on a simple object on the server-side code. If we reboot the server we will lose these values.

  \begin{block}{Simple object lost when server restarts}
    {\scriptsize
    \begin{verbatim}
    var playerPositions = {
        player1: { x: 100, y: 200 },
        player2: { x: 100, y: 400 },
        player3: { x: 700, y: 200 },
        player4: { x: 700, y: 400 }
    };
    \end{verbatim}
    }
  \end{block}
\end{frame}

\begin{frame}[fragile]
  \frametitle{RedisDB}
  \begin{block}{Website definition}
    Redis is an open source, BSD licensed, advanced key-value store. It is often referred to as a data structure server since keys can contain strings, hashes, lists, sets and sorted sets.
  \end{block}

  \begin{center}
    \includegraphics[width=100px]{images/redis-logo.jpg}
  \end{center}
\end{frame}

\begin{frame}[fragile]
  \frametitle{RedisDB: Installation}

  \begin{block}{From source code}
    {\tiny
    \begin{verbatim}
    $ wget http://download.redis.io/redis-stable.tar.gz
    $ tar xvzf redis-stable.tar.gz
    $ cd redis-stable
    $ make
    \end{verbatim}
    }
  \end{block}

  \pause

  \begin{block}{Copy executable and default configuration}
    {\tiny
    \begin{verbatim}
    $ sudo cp src/redis-server /usr/local/bin/ 
    $ sudo cp src/redis-cli /usr/local/bin/ 
    $ sudo cp redis.conf /etc/redis.conf
    \end{verbatim}
    }
  \end{block}

  \pause

  \begin{block}{Run server}
    {\tiny
    \begin{verbatim}
    $ redis-server
    [28550] 01 Aug 19:29:28 # Warning: no config file specified, using the default config...
    [28550] 01 Aug 19:29:28 * Server started, Redis version 2.2.12
    [28550] 01 Aug 19:29:28 * The server is now ready to accept connections on port 6379
    ...
    \end{verbatim}
    }
  \end{block}
\end{frame}

\begin{frame}[fragile]
  \frametitle{RedisDB: Client}

  \begin{block}{Installation}
    {\tiny
    \begin{verbatim}
    npm install redis --save-dev
    \end{verbatim}
    }
  \end{block}

  \pause

  \begin{block}{Create client on our code}
    {\tiny
    \begin{verbatim}
    var redis = require('redis'),
        redisClient = redis.createClient();
    \end{verbatim}
    }
  \end{block}
\end{frame}

\begin{frame}[fragile]
  \frametitle{RedisDB: Commands}
   In this workshop we are going to use one Redis data structure: a Hash

   \pause

   \begin{block}{Get all fields and values from a hash}
     {\tiny
     \begin{verbatim}
     redisClient.hgetall('myHash', function (err, result) {
         // result is converted on a Javascript object
     });
     \end{verbatim}
     }
   \end{block}

   \begin{block}{Store some fields and values to a hash}
     {\tiny
     \begin{verbatim}
     redisClient.hmset(['myHash', 'key1', 'value1', 'key2', 'value2'], function (err, result) {
        // code omitted
     });
     \end{verbatim}
     }
   \end{block}
\end{frame}

\begin{frame}[fragile]
  \frametitle{RedisDB: Lab}

  \begin{itemize}
    \item \texttt{git checkout stage\_9}
    \item Install your back-end dependencies with \texttt{npm install}
    \item Install your front-end dependencies with \texttt{bower install}
    \item Start \texttt{grunt watch} for auto linting
    \item Find TODOs and complete the exercise.
  \end{itemize}
\end{frame}
