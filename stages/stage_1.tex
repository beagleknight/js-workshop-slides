\section{Stage 1: Introduction}

\begin{frame}
\end{frame}

\begin{frame}
  \frametitle{Features}

  \begin{itemize}
    \item Loosely typed language
    \pause\item Object literal notation
    \pause\item Prototypal inheritance
    \pause\item Global variables
    \pause\item Functions are first class objects
  \end{itemize}
\end{frame}

\begin{frame}
  \frametitle{Features}

  \begin{block}{ECMAScript}
    The standard that defines JavaScript is the third edition of \textit{ECMAScript Programming Language}.
  \end{block}
\end{frame}

\begin{frame}[fragile]
  \frametitle{Hello World}

  \begin{block}{index.html}
    {\scriptsize
    \begin{verbatim}
    <html>
      <head>
        <script>
          document.writeln(``Hello, world!'');
        </script>
      </head>
      <body>
      </body>
    </html>
    \end{verbatim}
    }
  \end{block}
\end{frame}

\begin{frame}[fragile]
  \frametitle{Syntax}

  \begin{block}{Comments}
  Block comments formed with /* */ and line-ending comments starting with //. Example:
  {\scriptsize
  \begin{verbatim}
  /* 
    We are learning Javascript and comments are very important
  */
  document.writeln(``Hello World!''); // Output: Hello World!
  \end{verbatim}
  }
  \end{block}

  \pause
  
  \begin{block}{Names}
    Starts with a letter or underscore and optionally followed by on or more letters, digits or underscores. Beware of some reserved words.
    {\scriptsize
    \begin{verbatim}
    bullet           // valid     _mana            // valid
    3force 	         // invalid   lucky42          // valid
    rocket-launcher  // invalid   grenade_launcher // valid
    \end{verbatim}
    }
  \end{block}
\end{frame}

\begin{frame}[fragile]
  \frametitle{Syntax}

  \begin{block}{Numbers}
    Single number type represented internally as 64-bit floating point.
    {\scriptsize
    \begin{verbatim}
    42 
    3.141516
    10e5
    1/0 // Output: Infinity
    0/0 // Output: NaN
    \end{verbatim}
    }
  \end{block}

  \pause

  \begin{block}{Strings}
    Can be wrapped in single quotes or double quotes. It can contains 0 or more characters. All characters in Javascript are 16 bits wide.
    {\scriptsize
    \begin{verbatim}
    ``Hello World''
    'Hello World'
    ``This is\n a multiline string''
    'You can write `` on single quotes string'
    \end{verbatim}
    }
  \end{block}
\end{frame}

\begin{frame}[fragile]
  \frametitle{Syntax}

  \begin{block}{Functions}
  {\scriptsize
  \begin{verbatim}
  function helloWorld (name) {
      console.log('Hello ' + name + '!');
  }

  helloWorld('David'); // Output 'Hello David!'

  var myFunction = function () {
      console.log('Hi there!');
  };

  myFunction(); // Output: 'Hi there!'
  \end{verbatim}
  }
  \end{block}
\end{frame}

\begin{frame}[fragile]
  \frametitle{Syntax}

  \begin{block}{Variables}
    Use the \texttt{var} keyword followed by a name to declare a variable. When used inside of a function, the \texttt{var} statement defines the function's private variables.
    {\scriptsize
    \begin{verbatim}
    var player; // variable player declared on a global scope

    function test() {
        var enemy; // Scoped to function test
    }
    \end{verbatim}
    }
  \end{block}
\end{frame}

\begin{frame}[fragile]
  \frametitle{Syntax}

  \begin{block}{Strict (in)equality}
  {\scriptsize
  \begin{verbatim}
  10 == '10' // Output: true, auto type coercion
  10 === '10' // Output: false strict equality
  10 != '10' // Output: false, auto type coercion
  10 !== '10' // Output: true strict inequality
  \end{verbatim}
  }
  \end{block}

  \pause

  \begin{block}{\texttt{null} and \texttt{undefined}}
  {\scriptsize
  \begin{verbatim}
  console.log(mario); // Error: mario is not defined

  function exists (mario) {
    console.log(mario);
  }

  exists(); // Output undefined

  console.log(null == undefined) // Output: true 
  console.log(null === undefined) // Output: false 
  \end{verbatim}
  }
  \end{block}
\end{frame}

\begin{frame}[fragile]
  \frametitle{Syntax}

  \begin{block}{\texttt{if}, \texttt{else}}
    \scriptsize{
    \begin{verbatim}
    var testOk = true;

    if (testOk) {
        console.log(``Captain obvious'');
    } else {
        console.log(``I'm bored'');
    }
    \end{verbatim}
    }
    Here are the \textit{falsy} values:
    \begin{itemize}
      \item \texttt{false}
      \item \texttt{null}
      \item \texttt{undefined}
      \item The empty string
      \item The number 0
      \item The number NaN
    \end{itemize}
    All other values are \textit{truthy}.
  \end{block}
\end{frame}

\begin{frame}[fragile]
  \frametitle{Syntax}

  \begin{block}{\texttt{switch}}
    \scriptsize{
    \begin{verbatim}
    var weapon = ``rocketlauncher'';

    switch(weapon) {
        case ``pistol'':
            console.log(``piu piu'');
            break;
        case ``shotgun'':
            console.log(``paaam!'');
            break;
        case ``rocketlauncher''
            console.log(``BOOOOM!'');
            break;
        default:
            console.log(``falcon punch!'');
            break;
    }
    \end{verbatim}
    }
  \end{block}
\end{frame}

\begin{frame}[fragile]
  \frametitle{Syntax}

  \begin{block}{\texttt{while}, \texttt{do while}}
    \scriptsize{
    \begin{verbatim}
    var counter = 0;
    while (counter < 10) { // Ends when counter is equal to 10
        console.log(counter);
        counter += 1;
    }

    do {
        console.log(counter);
        i -= 1;
    } while(counter > 0); // Ends when counter is equal to 0
    \end{verbatim}
    }
  \end{block}

  \pause

  \begin{block}{\texttt{for}}
    {\scriptsize
    \begin{verbatim}
    var i;

    for (i = 0; i < 10; i += 1)
        console.log(i);
    }
    \end{verbatim}
    }
  \end{block}
\end{frame}
