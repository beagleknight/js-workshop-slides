\section{Stage 1: Introduction}

\begin{frame}
  \frametitle{Stage 1: Introduction}
  Features:
  \begin{itemize}
    \item Loosely typed language
    \item Object literal notation
    \item Prototypal inheritance
    \item Global variables
    \item Functions are first class objects
  \end{itemize}
\end{frame}

\begin{frame}
  \begin{block}{ECMAScript}
    The standard that defines JavaScript is the third edition of \textit{ECMAScript Programming Language}.
  \end{block}
\end{frame}

\begin{frame}[fragile]
  \frametitle{Hello World}

  \begin{block}{index.html}
    {\scriptsize
    \begin{verbatim}
    <html>
      <head>
        <script>
          document.writeln(``Hello, world!'');
        </script>
      </head>
      <body>
      </body>
    </html>
    \end{verbatim}
    }
  \end{block}
\end{frame}

\begin{frame}[fragile]
  \begin{block}{Comments}
  Block comments formed with /* */ and line-ending comments starting with //. Example:
  {\scriptsize
  \begin{verbatim}
  /* 
    We are learning Javascript and comments are very important
  */
  document.writeln(``Hello World!''); // Output: Hello World!
  \end{verbatim}
  }
  \end{block}
  
  \begin{block}{Names}
    Starts with a letter or underscore and optionally followed by on or more letters, digits or underscores. Beware of some reserved words.
    {\scriptsize
    \begin{verbatim}
    bullet           // valid     _mana          // valid
    weapon           // valid     life_          // valid
    3force 	         // invalid   lucky42        // valid
    rocket-launcher  // invalid   Hammer         // valid
    grenade_launcher // valid     hammer         // valid, case sensitive
    \end{verbatim}
    }
  \end{block}
\end{frame}

\begin{frame}[fragile]
  \begin{block}{Numbers}
    Single number type represented internally as 64-bit floating point.
    {\scriptsize
    \begin{verbatim}
    1
    1.0
    3.141516
    10e5
    -5E-10
    .123456
    1/0 // Output: Infinity
    0/0 // Output: NaN
    \end{verbatim}
    }
  \end{block}

  \begin{block}{Strings}
    Can be wrapped in single quotes or double quotes. It can contains 0 or more characters. All characters in Javascript are 16 bits wide.
    {\scriptsize
    \begin{verbatim}
    ``Hello World''
    'Hello World'
    ``This is\n a multiline string''
    'You can write `` on single quotes string'
    \end{verbatim}
    }
  \end{block}
\end{frame}

\begin{frame}[fragile]
  \begin{block}{Variables}
    Use the \texttt{var} keyword followed by a name to declare a variable. When used inside of a function, the \texttt{var} statement defines the function's private variables.
    \scriptsize{
    \begin{verbatim}
    var player; // variable player declared on a global scope

    function test() {
        /* 
            Variable enemy is visible on this function but 
            we have access to variable player.
        */
        var enemy;

        function test2() {
            /* 
                Variable bullet is visible on this function but 
                we have access to player and enemy variables.
            */
            var bullet; 
        }
    }
    \end{verbatim}
    }
  \end{block}
\end{frame}

\begin{frame}[fragile]
  \begin{block}{\texttt{if}, \texttt{else}}
    \scriptsize{
    \begin{verbatim}
    var testOk = true;

    if (testOk) {
        console.log(``Captain obvious'');
    } else {
        console.log(``I'm bored'');
    }
    \end{verbatim}
    }
    Here are the \textit{falsy} values:
    \begin{itemize}
      \item \texttt{false}
      \item \texttt{null}
      \item \texttt{undefined}
      \item The empty string
      \item The number 0
      \item The number NaN
    \end{itemize}
    All other values are \textit{truthy}.
  \end{block}
\end{frame}

\begin{frame}[fragile]
  \begin{block}{\texttt{switch}}
    \scriptsize{
    \begin{verbatim}
    var weapon = ``rocketlauncher'';

    switch(weapon) {
        case ``pistol'':
            console.log(``piu piu'');
            break;
        case ``shotgun'':
            console.log(``paaam!'');
            break;
        case ``rocketlauncher''
            console.log(``BOOOOM!'');
            break;
        default:
            console.log(``falcon punch!'');
            break;
    }
    \end{verbatim}
    }
  \end{block}
\end{frame}

\begin{frame}[fragile]
  \begin{block}{\texttt{while}, \texttt{do while}}
    \scriptsize{
    \begin{verbatim}
    var counter = 0;

    while (counter < 10) { // Ends when counter is equal to 10
        console.log(counter);
        counter += 1;
    }

    do {
        console.log(counter);
        i -= 1;
    } while(counter > 0); // Ends when counter is equal to 0
    \end{verbatim}
    }
  \end{block}

  \begin{block}{\texttt{for}}
    \scriptsize{
    \begin{verbatim}
    var i;

    for (i = 0; i < 10; i += 1)
        console.log(i);
    }
    \end{verbatim}
    }
  \end{block}
\end{frame}

\begin{frame}[fragile]
  \begin{block}{Literals}
    \scriptsize{
    \begin{verbatim}
    var score = 9000;                                   // Number
    var name = ``David'';                               // String
    var weapons = [``Pistol'', ``Shotgun'', ``Sword'']; // Array
    var player = {                                      // Object
        name: ``Megaman'',
        lifes: 3,
        weapons: [``Buster'', ``Laser'']
    };
    var run = function (param1, param2) {               // Function
    };
    var validation = /^pro/;                            // Regexp
    \end{verbatim}
  \end{block}
\end{frame}

\begin{frame}[fragile]
  \frametitle{Objects}
  \begin{itemize}
    \item Objects in Javascript are mutable keyed collections.
    \item Arrays, functions and regular expressions are objects.
    \item A property name can be any string.
    \item Objects can inherit properties of another through its prototype.
  \end{itemize}
  \begin{block}{\texttt{Prototype}}
    Every object is linked to a prototype object form which it can inherit properties. All objects created from object literals are linked to \texttt{Object.prototype}.

    The prototype link is used only in retrieval. If we try to retrieve a property value from an object, and if the object lacks the property name, then Javascript attempts to retrieve the property value from the prototype object.
  \end{block}
\end{frame}
