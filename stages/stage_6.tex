\section{Stage 6: Refactor our client-side code with Require.js}

\begin{frame}[fragile]
  \frametitle{Problem}

  \begin{block}{Including scripts}
    {\scriptsize
    \begin{verbatim}
    <html>
      <head>
        <script src=``js/game.js''></script>
        <script src=``js/character.js''></script>
        <script src=``js/player.js''></script>
        <script src=``js/enemy.js''></script>
        <script src=``js/knight.js''></script>
        <script src=``js/soldier.js''></script>
        <script src=``js/protector.js''></script>
        ...
    \end{verbatim}
    }
  \end{block}

  \pause
  
  \begin{itemize}
    \item We need to remember the inclusion order 
    \item Each module, function or object must be accessible through global scope if we want to use it as a dependency.
  \end{itemize}
\end{frame}

\begin{frame}[fragile]
  \frametitle{Possible solution}

  \begin{block}{index.html}
    {\tiny
    \begin{verbatim}
    <html>
      <head>
        <script src=``js/built.js''></script>
        ...
    \end{verbatim}
    }
  \end{block}

  \begin{block}{Gruntfile.js}
    {\tiny
    \begin{verbatim}
    ...
        concat: {
            options: {
                separator: ';',
            },
            dist: {
                src: [
                    'js/game.js',
                    'js/character.js',
                    'js/player.js',
                    'js/enemy.js',
                    'js/knight.js',
                    'js/soldier.js',
                    'js/protector.js'
                ],
    ...
    \end{verbatim}
    }
  \end{block}
\end{frame}

\begin{frame}[fragile]
  \frametitle{Require.js}

  \begin{block}{A javascript module loader}
    RequireJS is a JavaScript file and module loader. It is optimized for in-browser use, but it can be used in other JavaScript environments, like Rhino and Node. Using a modular script loader like RequireJS will improve the speed and quality of your code.
  \end{block}

  \begin{center}
    \includegraphics[width=100px]{images/requirejs-logo.png}
  \end{center}
\end{frame}

\begin{frame}[fragile]
  \frametitle{Require.js: an example}

  \begin{columns}[c]

  \column{2.25in}
  \begin{block}{Without Require.js}
    {\scriptsize
    \begin{verbatim}
    var MYGAME = MYGAME || {},
        game   = MYGAME.game,
        entity = MYGAME.entity;

    MYGAME.crate = function (spec) {
      // code omitted
    };
    \end{verbatim}
    }
  \end{block}

  \column{2.25in}

  \begin{block}{With Require.js}
    {\scriptsize
    \begin{verbatim}
    define(function (require) {
      var game   = require('game'),
          entity = require('entity'),
          crate;

      crate = function (spec) {
      };

      return crate;
    });
    \end{verbatim}
    }
  \end{block}

  \end{columns}
\end{frame}

\begin{frame}[fragile]
  \frametitle{Require.js: getting started}

  \begin{block}{Get Require.js}
  Use \texttt{bower} to install it as a dependency of your project
  \end{block}

  \pause

  \begin{block}{Include it}
    {\tiny
    \begin{verbatim}
    <html>
      <head>
        <script data-main=``scripts/main'' src=``bower_components/requirejs/require.js''></script>
        ...
    \end{verbatim}
    }
  \end{block}

  \pause

  \begin{block}{Define modules}
    {\scriptsize
    \begin{verbatim}
    define(function (require) {
        // code omitted
    });
    \end{verbatim}
    }
  \end{block}
\end{frame}

\begin{frame}[fragile]
  \frametitle{Require.js: Lab}
  \begin{block}{Exercise}
    \begin{itemize}
      \item \texttt{git checkout stage\_6}
      \item Install your back-end dependencies with \texttt{npm install}
      \item Install your front-end dependencies with \texttt{bower install}
      \item Start \texttt{grunt watch} for auto linting
      \item Install Require.js and include the main entry point
      \item Refactor modules and functions using Require.js modules.
    \end{itemize}
  \end{block}
\end{frame}
